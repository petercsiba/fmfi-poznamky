\documentclass[12pt,a4paper]{article}

\usepackage[slovak]{babel}
\usepackage[utf8]{inputenc}
\usepackage[T1]{fontenc}

\usepackage{listings}
\usepackage{graphicx}
\usepackage{tabularx} 

\usepackage{hyperref} 

\usepackage{amsmath} 

\lstset{
language=sh
,breaklines=true
,basicstyle=\ttfamily
, showstringspaces=false}

\author{Peter Csiba}
\textwidth 6.5in
\oddsidemargin 0.0in
\evensidemargin 0.0in

\title{Pravdepodobnostné metódy - Domáca úloha \#4 - UNIQUE-CLIQUE}
\date{10-04-2014}
\author{Peter Csiba, csiba4@uniba.sk}

\begin{document}
\maketitle

\section*{Zadanie}
Hľadanie najväčšej kliky v grafe je NP-ťažký problém. Presnejšie, CLIQUE je nasledovný rozhodovací problém: pre daný graf G a číslo k - existuje klika veľkosti k v G? Tento problém je NP-úplný. Uvažujme teraz pozmenený problém UNIQUE-CLIQUE, v ktorom nás opäť zaujíma, či existuje klika veľkosti k v G, avšak máme zaručené, že taká klika buď neexistuje, alebo je práve jedna. (Algoritmus riešiaci túto úlohu musí odpovedať správne na vstupoch s 0 alebo 1 klikou veľkosti k; na ostatných vstupoch môže dať akúkoľvek odpoveď.) Dokážte, že toto nám veľmi nepomôže a problém UNIQUE-CLIQUE nie je oveľa ľahší ako CLIQUE. Presnejšie: ak by existoval polynomiálny algoritmus riešiaci UNIQUE-CLIQUE, tak existuje randomizovaný polynomiálny algoritmus riešiaci CLIQUE.

\paragraph{Riešenie.}
Na ďalšej strane. 

\section*{Zdroje} 
\begin{itemize} 
\item \url{http://en.wikipedia.org/wiki/Isolation_lemma} 
\item \url{http://http.cs.berkeley.edu/~vazirani/pubs/matching.pdf} 
\end{itemize} 

\newpage
\section*{Riešenie}
Nasledovný algoritmus s pravdepodobnosťou väčšou ako 1/2 a využitím UNIQUE-CLIQUE odpovie či $G(N,k) \in CLIQUE$:

\begin{lstlisting} 
(1) Ovahuj vrcholy w(v) nahodnymi vahami z <1,2n> 
(2) Vytvor G' kde kazdy vrchol v' je klikou velkosti n^3 + w(v) 
(3)   Pospajaj u' v' kompletnym bipartitnym grafom prave vtedy ged u-v je hranou v G
(4) for r from 2nk to n:
(5)   if (G',kn^2 + r) in UNIQUE-CLIQUE:
(6)     return (G,k) in CLIQUE
(7) return (G,k) not in CLIQUE
\end{lstlisting} 

Pre (1) z \emph{Isolation lema} od (Mulmuley, Vazirani, and Vazirani) platí, že s pravdepodobnosťou väčšou ako $1/2$ bude klika s najväčšou váhou jediná v grafe. Keďže máme len algoritmus pre neváhované grafy, tak potrebujem transformovať $G$ na $G'$ tak, aby algoritmus UNIQUE-CLIQUE našiel kliku veľkosti $k$ s najväčšou váhou. Vidíme, že ak mal graf $G$ kliku $K$ veľkosti $k$ s celkovou váhou $x$ tak $G'$ má kliku veľkosti $kn^2 + x$, lebo všetky vrcholy $K$ sú navzájom poprepájané v $G'$ a každý je veľkosti $n^2 + w(v)$. Takže najväčšia klika $G'$ bude jedinou najväčšou klikou $G'$ s pravdepodobnosťou väčšou ako $1/2$. 
Keby $(G,k) \notin CLIQUE$ tak stačí nájsť najväčšiu kliku v $G'$. Akú môže mať najväčšiu možnú veľkosť? Predpokladajme, že $k$ je najväčšia klika v $G$ (ak nie, tak algoritmus môžeme spustiť najprv pre $n, n-1, \ldots, k+1$). Potom keby každý vrchol z $K$ mal maximálnu váhu $2n$ tak v $G'$ to znamená $kn^2 + 2nk$. 
Keby $(G,k) \notin CLIQUE$ a tak náš algoritmus odpovie \emph{nie}, pretože súčet priradených váh nestačí na prekonanie $n^3$ veľkostí vrcholov. 
Aká je časová zložitosť? Generovanie grafu $G'$ zaberie $O(n^8)$ a UNIQUE-CLIQUE spúšťame maximálne $n-k+1$ krát na grafe veľkosti $O(n^4)$. Celý náš pravdepodobnostný algortimus je preto polynomiálny.  

\end{document}

