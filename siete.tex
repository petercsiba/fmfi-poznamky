\documentclass[10pt,a4paper]{article}
\usepackage[slovak]{babel}
\usepackage[utf8]{inputenc}
\usepackage{amsmath}
\usepackage{amsfonts}
\usepackage{amssymb}
\usepackage[unicode]{hyperref}
\usepackage{graphicx}

\textwidth 6.5in
\oddsidemargin 0.0in
\evensidemargin 0.0in

\title{Poznámky z počítačových sietí - matroš na štátnice}
\date{15.06.2012}
\author{Peter Csiba, petherz@gmail.com, \url{https://bitbucket.org/petrzlen/fmfi_poznamky}} 

\begin{document}
\maketitle
\tableofcontents

\clearpage

%%%%%%%%%%%%%%%%%%%%%%%%%%%%%%%%%%%%%%%%%%%%%%%%%%%%%%%%%%%%%%%%%%%%%%%%%%%%%%%%
\paragraph{Úvod.}   

Text sú poznámky k oficiálnym \href{štátnicovým otázkam}{http://new.dcs.fmph.uniba.sk/index.php/Studium/Bakalarske/StatneSkusky} a boli spísané počas učenia sa na ne.
Poznámky sa nesnažia ísť do hĺbky (na to je Tanenbaum, Wikipédia a RFC), naopak, snažia sa priniesť intuitívnu predstavu o technológiách a dávať jednotlivé pojmy do súvisu. 
Autor neabsolvoval prednášky ani test z predmetu Počítačové siete. Pán RNDr. Jaroslav Janáček PhD. ohodnotil znalosti počítačových sietí autora pred písaním na D+\footnote{
Čo autora motivovalo naučiť sa to poriadne. 
}, takže autor preto neručí za korektnosť textu.
Poznamenajme, že autor sa snažil písať pravdu a len pravdu, keďže jeho odpoveď na štátniciach vychádza z tototo materiálu. V skratke, čitateľ informácie z tohoto textu vstrebáva na vlastné riziko :).

Autor odporúča si pozrieť detaily hlavných protokolov, ako IP, UDP, TCP, SSL/TLS alebo DNS. 

%%%%%%%%%%%%%%%%%%%%%%%%%%%%%%%%%%%%%%%%%%%%%%%%%%%%%%%%%%%%%%%%%%%%%%%%%%%%%%%%
\section{Telekomunikácie}           
%%%%%%%%%%%%%%%%%%%%%%%%%%%%%%%%%%%%%%%
\subsection{Koncept prenosu informácií} 
???

%%%%%%%%%%%%%%%%%%%%%%%%%%%%%%%%%%%%%%%
\subsection{Integrácia}
???
               
%%%%%%%%%%%%%%%%%%%%%%%%%%%%%%%%%%%%%%%
\subsection{MPLS (Multiprotocol Label Switiching)}
Medzi 2. a 3. vrstvou OSI ("layer 2.5" protocol). 
Podporuje priame spojenia, aj circuit based (na viacero skokov).
Hlavička (label) sa skladá z adresy na linkovej vrstve. Hlavičky môžu tvoriť stack, čím sa umožňuje posielať datagram cez viacero liniek.

Navrhnutý 1996, podpora pre väčšinu linkových a dátových protokolov. Unifikuje prístup k nim. 

%%%%%%%%%%%%%%%%%%%%%%%%%%%%%%%%%%%%%%%%%%%%%%%%%%%%%%%%%%%%%%%%%%%%%%%%%%%%%%%%
\section{Základné pojmy zo sietí}    
%%%%%%%%%%%%%%%%%%%%%%%%%%%%%%%%%%%%%%%
\subsection{Topológia a geografia} 
Topológia - organizácia siete. Fyzická (aká je v skutočnosti) a logická (ako sa správa). Nemusia byť rovnaké, napr. zariadenia spojené Ethernetom a hubom majú fyzickú topológiu hviezdy a logickú topológiu zbernice (lebo posielaný signál počuje každý).  
\begin{itemize}
\item Point-to-point - dve zariadenia spojené priamo.         
\item Zbernica (bus) - viaceré zariadenia pripojené na zdielané médium. Ethernet spojený hubom, odpočúvanie telefónnych liniek alebo zbernice vnútri počítača. Na riadenie komunikácie sa používa CSMA/CD alebo CSMA/CA (collision detection a collision avoidance s exponential backoff).
\item Hviezda (star) - centralizovaný prvok riadi komunikáciu. Každé zariadenie je spojené so šéfom štýlom point-to-point. Napr. router.  Výhodou je, že je triviálne pridať nové zariadenie. 
\item Kruh (ring) - do kruhu. Obľúbený teoretický model. Signály sa posielajú v dohodnutom smere. 
\item Mesh - see WiFi. Každý má rovnakú úlohu. 
\item Kompletne spojený - Každý s každým (zjavná nevýhoda: počet spojení rastie kvadraticky). 
\item Strom - hierarchická štruktúra, je daný koreň stromu, medzi vrcholmi sú point-to-point (graf netvorí cyklus). Môže na ňom byť implementované smerovanie (routing). 
\item Hybrid - napr. Internet. 
\item Daisy chain - spojenie viacerých zberníc, ktoré môžu tvoriť cyklus. 
\end{itemize}

Kategorizácia podľa geografie siete (od najmenších po najväčšie):
\begin{itemize}
\item PAN - Personal area network - do 10m. USB, Bluetooth, LAN párty.           
\item LAN - Local area network - jedna budova, škola, internát. Typicky Ethernetové káble.  
\item Chrbtica - Backbone - spája viacere menšie siete (podsiete - \emph{subnets}). Špeciálne prípad: Internet backbone.
\begin{center}
\includegraphics[scale=0.4]{backbone.jpg}
\end{center} 
\item WAN - Wide area network - mestá, štáty až medzikontinentálne. 
\item Internetwork - prepojenie viacerých sietí (napr. nackbone-om).
\end{itemize}
Niektoré sme vynechali. 

%%%%%%%%%%%%%%%%%%%%%%%%%%%%%%%%%%%%%%%
\subsection{Základné typy sietí, informačné toky}  
???
\begin{itemize}
\item Informačné 
\item Telekomunikačné
\item Sociálne
\item Neurónové, Náhodné,...            
\end{itemize}

%%%%%%%%%%%%%%%%%%%%%%%%%%%%%%%%%%%%%%%
\subsection{Zdroje, cieľové uzly, prepínací systém}
???
Asi niečo o sieťových zariadeniach. 
 
%%%%%%%%%%%%%%%%%%%%%%%%%%%%%%%%%%%%%%%
\subsection{Jednosmerné a obojsmerné spojenia}
       
\begin{itemize}                                  
\item Obojsmerné (full-duplex). Oboma smermi paralelne, ako mobilné telefóny.
\item Jednosmerné (half-duplex). Len jedným smerom naraz, ako policajné vysielačky.
\end{itemize} 
     
%%%%%%%%%%%%%%%%%%%%%%%%%%%%%%%%%%%%%%%
\subsection{Konferencie, komunikačné kanály}  
???
Dosť všeobecné. 
Konferenčné hovory boli implementované aj na telekomunikačných linkách. Jedným spôsobom boli špeciálne zariadenia - mosty - ktorým sa priradilo virtuálne telefónne číslo a na ňom sa zdielala komunikácia. Druhým spôsobom bolo (napr. v UK) pridanie špeciálneho tlačítka, ktoré umožňovalo zavolať tretiemu účastníkovy od druhého (predĺžiť tak spojenie $A \rightarrow B$ na $A \rightarrow B \rightarrow C$. 
  
%%%%%%%%%%%%%%%%%%%%%%%%%%%%%%%%%%%%%%%
\subsection{Multiplexovanie}   
\label{multiplexing}
Viacero komunikácií (protokolov, pagáčov makových) cez jednu linku. 
Napr. sťahovanie torrentov a pozeranie emailov naraz (resp. hranie age-a). 
Viac na \href{http://en.wikipedia.org/wiki/Multiplexing}{Wikipédií}.
       
%%%%%%%%%%%%%%%%%%%%%%%%%%%%%%%%%%%%%%%
\subsection{Virtuálne okruhy (virtual circuit)}     
Na sieťovej vrstve sa vytvorí cesta (obojsmerná) cez zariadenia, po ktorej sa posielajú dáta (bity, signály, fotóny, ...) medzi koncovými vrcholmi. Vytvorenie spojenia zaručuje, že dáta prídu v rovnakom poradí, ako boli odoslané. Opakom sú datagramy (napr. UDP), ktoré každý kus dát posielajú cez sieť (routre) neurčito.   
    
%%%%%%%%%%%%%%%%%%%%%%%%%%%%%%%%%%%%%%%%%%%%%%%%%%%%%%%%%%%%%%%%%%%%%%%%%%%%%%%%
\section{Schéma jednoduchého komunikačného modelu, sieťový software} 
???

%%%%%%%%%%%%%%%%%%%%%%%%%%%%%%%%%%%%%%%
\subsection{Technika štrukturovaného sieťového softwaru} 
???
  
%%%%%%%%%%%%%%%%%%%%%%%%%%%%%%%%%%%%%%%
\subsection{Koncepcia vrstiev, protokolov a interface} 
Pozri \ref{OSI}. 

%%%%%%%%%%%%%%%%%%%%%%%%%%%%%%%%%%%%%%%
\subsection{Virtuálna a fyzická komunikácia}      
???
Buď ju pozvem na večeru (fyzická), alebo si chatujem na fejsbúčiku (virtuálna - využíva viaceré spojenia na dosiahnutie point-to-point). 
                            
%%%%%%%%%%%%%%%%%%%%%%%%%%%%%%%%%%%%%%%%%%%%%%%%%%%%%%%%%%%%%%%%%%%%%%%%%%%%%%%%
\section{Všeobecné závery z oblasti počítačových sietí, ktoré musia byť zakomponované do vrstvovej sieťovej architektúry}   
%%%%%%%%%%%%%%%%%%%%%%%%%%%%%%%%%%%%%%%
\subsection{Adresovanie}    
Identifikovanie zariadení v rámci siete. Plochá addresácia (flat addressing) - adresy sa prideľujú inkrementovaním countera (alebo tomu izomorfným spôsobom), napr. MAC adresy.
Hierarchická adresácia (hierarchical addressing) - tvoria strom, napr. IP adresy (IANA prideľuje adresy kontinentom a štátom, tie prideľujú adresy ISP (Internet Service Provider, napr. Orange, T-com, ...),...) alebo telefónne čísla (keď ešte fungovali cez switche).
              
%%%%%%%%%%%%%%%%%%%%%%%%%%%%%%%%%%%%%%%
\subsection{Pravidlá pre prenos údajov}   
???
Prirovnáme to k cestnej doprave. 

%%%%%%%%%%%%%%%%%%%%%%%%%%%%%%%%%%%%%%%
\subsection{Správa chýb}
Integrita správ (checksumy), samoopravné kódy (\href{http://www.dcs.fmph.uniba.sk/texty/codebook.pdf}{Teória kódovania skriptá}). 
Robí sa to skoro na každej úrovni (hlavička aj telo (payload)).
                  
%%%%%%%%%%%%%%%%%%%%%%%%%%%%%%%%%%%%%%%
\subsection{Postupnosť (následnosť) správ}
V prípade point-to-point nie je čo riešiť. Ak sa dáta smerujú (routujú) cez viacero zariadení, tak nemusí byť zabezpečená, keďže môžu ísť rôznymi cestami (napr. keď sa router rozhodne predísť upchatiu na ďalšej linke). 
Ak je vytvorený virtuálny obvod, tak je postupnosť zabezpečená. Inak napr. TCP (\ref{TCP}).
 
%%%%%%%%%%%%%%%%%%%%%%%%%%%%%%%%%%%%%%%
\subsection{Problém rýchleho odosielateľa a pomalého príjemcu}
Pozri \href{http://en.wikipedia.org/wiki/Flow\_control}{Flow control}.
Rieši to napr. TCP pomocou posuvných okien (sliding windows, viac \ref{TCP}).
Problém je inštanciou problémov upchania siete (môže byť dôsledkom útoku),
presnejšie sa snaží upchaniu predchádzať (avoidance) ako riešiť vzniknuté (detection).
  
%%%%%%%%%%%%%%%%%%%%%%%%%%%%%%%%%%%%%%%
\subsection{Neschopnosť akceptovať správy ľubovoľnej dĺžky}  
Napr. $\infty$ dĺžky :)
Konkrétnym prípadom je získanie MTU (maximum transport unit) v IP paketoch. 
Nie všetky zariadenia vedia spracovať ľubovoľne dlhý paket. Napr. IPv4 umožňuje pakety sekať - \emph{fragmentácia paketov}, IPv6 definuje protokol na získanie MTU pre danú cestu a ponecháva na klientovi, aby zistil a dodržal túto hodnotu (inak bude jeho paket aj tak zahodený).

%%%%%%%%%%%%%%%%%%%%%%%%%%%%%%%%%%%%%%%
\subsection{Efektívny prenos malých správ} 
???
V dnešnej dobe, keď sa jeden klik klávesy pri SSH prenáša pomocou HTTP paketu?
(Asi 1000 násoboný overhead).
      
%%%%%%%%%%%%%%%%%%%%%%%%%%%%%%%%%%%%%%%
\subsection{Multiplexovanie a demultiplexovanie}  
Pozri \ref{multiplexing}. Demultiplexovanie je inverznou operáciou. 

%%%%%%%%%%%%%%%%%%%%%%%%%%%%%%%%%%%%%%%
\subsection{Smerovanie (routing)}   
Široká téma \href{http://en.wikipedia.org/wiki/Routing}{Routing}. 
Popísané v \ref{routing}.
              
                                                         
%%%%%%%%%%%%%%%%%%%%%%%%%%%%%%%%%%%%%%%%%%%%%%%%%%%%%%%%%%%%%%%%%%%%%%%%%%%%%%%%
\section{Rozhrania a služby}                
Sieťová služba (network service) - služby dostupné na sieti pre počítače pripojené na sieť. Často poskytujú zdielané zdroje a nazývajú sa aj port, daemon alebo listener.  K službe sa pristupuje transportnými protokolmi. 
Sieťové rozhranie (network interface controller) - používa sa na pripojenie k sieti. Napr. sieťová karta. Používajú aktívne čakanie na prijímanie paketov (Plachetka approved polling). 
     
%%%%%%%%%%%%%%%%%%%%%%%%%%%%%%%%%%%%%%%
\subsection{Vzťah medzi vrstvami a rozhraniami}  
???
Asi na akých vrstvách sieťové rozhrania komunikujú. Na najnižšej, zabezpečujú prenos dát na fyzickej vrstve.  

%%%%%%%%%%%%%%%%%%%%%%%%%%%%%%%%%%%%%%%
\subsection{Service Access Points (SAP's)} 
Koncové body v sieti, ktoré majú svoj identifikátor (asi).   
NSAP adresy v OSI sú podobné IP adresám. Identifikujú koncové body v sieti, ku ktorým sa pristupuje protokolom ATM (Asynchronous Transfer Mode, unifikuje prenos v dátovoých a telekomunikačných sieťach). ATM prenáša celly, ktoré majú fixnú veľkosť (narozdiel od rámcov a paketov v IP).  
  
%%%%%%%%%%%%%%%%%%%%%%%%%%%%%%%%%%%%%%%
\subsection{Interface Data Unit (IDU)}       
Pod týmto menom som to poriadne nenašiel. 
Tvári sa to ako gateway (z IP), ktorý premosťuje viaceré protokoly (konverzia dát) a 
umožňuje prístup k iným sieťam. 
      
%%%%%%%%%%%%%%%%%%%%%%%%%%%%%%%%%%%%%%%
\subsection{Spojované a nespojované služby}  
???
Nechápem. 


%%%%%%%%%%%%%%%%%%%%%%%%%%%%%%%%%%%%%%%
\subsection{Kvalita služby a službové primitíva}   
???
Náhodný generátor kľúčových atribútov kvality:
\begin{itemize}
\item Spoľahlivosť.
\item Bezpečnosť.
\item Rýchlosť.
\item Jednoduchosť.
\item Užitočnosť.
\end{itemize}
         
%%%%%%%%%%%%%%%%%%%%%%%%%%%%%%%%%%%%%%%%%%%%%%%%%%%%%%%%%%%%%%%%%%%%%%%%%%%%%%%%
\section{Referenčné modely}   
\label{reference_model}   
Všeobecná definícia \emph{referenčných modelov} v technickom svete: An abstract framework or domain-specific ontology consisting of an interlinked set of clearly defined concepts produced by an expert or body of experts in order to encourage clear communication. 
Laicky ide o všeobecné pravidlá architektúry systémov. Napr. \href{http://en.wikipedia.org/wiki/Von_Neumann_architecture}{von Neumannova architektúra} počítača (CPU, Pamäť, ...). Alebo napr. taký \emph{feng-shui} je referenčným modelom pre zariadenie domu. Pre každú miestnosť špecifikuje, ako by mala vyzerať (a hlavne ako nie) a tieto koncepty prepája medzi sebou (napr. záchod a kuchyňa by nemali susedieť). 
     
%%%%%%%%%%%%%%%%%%%%%%%%%%%%%%%%%%%%%%%
\subsection{Ciele a nebezpečenstvá}   
Náhodný generátor kľúčových atribútov cieľov a nebezpečenstiev (sú to protiklady):
\begin{itemize}
\item Jednoduchosť - ako v matematickej logike, chceme mať čo najmenej pravidiel. Ak budú globálne uznávané a používané, tak nie je problém systémy navrhnuté podľa referenčného modelu kombinovať, resp. je jednoduchšie pochopiť a navrhnúť nové systémy. 
\item Transparentnosť - aj fyzik, ktorý sa chce uživiť programovaním, by tomu mal pochopiť.
\item Užitočnosť - špecifikuje niečo reálne a väčšinou už používané.
\item Nebezpečenstvo - nedodržať ciele (napr. navrhnút tutti-frutti referenčný model, ktorý bude takým molochom, že sa tým nikto nebude riadiť). 
\end{itemize}

%%%%%%%%%%%%%%%%%%%%%%%%%%%%%%%%%%%%%%%
\subsection{Open Systems Architecture} 
Nemýliť si s OAS (Open Architecture Systems), používané Microsoftom, Citrix a VMware. 
Vrstvový referenčný model pre telekomunikácie. \href{www.acq.osd.mil/osjtf/pdf/boeing.pdf}{Jedna z implementácií} spoločnosťou Boeing. 
Presn/ špecifikáciu som nenašiel, zakladá sa na OSI. 

%%%%%%%%%%%%%%%%%%%%%%%%%%%%%%%%%%%%%%%
\subsection{Norma ISO 7498 (jedna z generácií OSI)}    
\label{OSI}

Určený pre všetky telekomunikačné siete a protokoly. 

Daná vrstva zahŕňa protokoly s podobnými funkcionalitami a cieľmi. Každá vrstva využíva len funkcie susedných vrstiev. 

\begin{itemize}
\item Aplikačná. Komunikácia medzi procesmi, majú špecializovanú úlohu (OSI definuje striktnejšie, IP zahŕňa aj Presentation a Session). DNS, FTP, HTTP, SMTP, DHCP. 
\item Prezentačná (občas nazývaná aj Syntax layer). Odbremeňuje koncové aplikácie od rôznych reprezentácií posielaných dát (znakových reťazcov),
napr. z dôvodu rôzneho kódovania koncových operačných systémov (ASCII, UTF, EBCDIC\footnote{
Extended Binary Coded Decimal Interchange Code (EBCDIC) is an 8-bit character encoding used mainly on IBM mainframe and IBM midrange computer operating systems.
}). Môže sa tu používať šifrovanie / dešifrovanie (aj keď nie nutne, napr. IPsec a VPN na sieťovej vrstve). SSL/TLS, MIME (Multipurpose Internet Mail Extensions).        
\item Session. Umožňuje koncovým aplikáciám (procesom) na koncových zariadeniach udržiavať stav spojenia (nie presne, pozri Session), napr. login na webstránkach. Využíva sa v RPC (Remote procedure calls - volanie procedúr koncových procesov) alebo pri streamovaní videa, aby zvuk nepredbiehal obraz a naopak. SOCKS, Named Pipe. 
\item Transportná. Sprístupnenie end-to-end komunikácie pre aplikácie, môže zabezpečovať spoľahlivosť, dátové prúdy a multiplexovanie. TCP, UDP.                    
\item Sieťová. Adresný priestor, smerovanie (routing) medzi koncovými stanicami. IP, ICMP, IPSec. 
\item Dátová. Vytváranie / posielanie základných logických jednotiek (postupnosti bitov) medzi fyzickými zariadeniami. Rámce (Frame), Switch, ATM (vytvára vituálne obvody)
\item Fyzická. Kabeláž. USB, Bluetooth, Wifi (IEEE 802.11).
\end{itemize}

Poznamenajme, že v skutočnosti je prvou hlavičkou hlavička fyzickej vrstvy. 
                  
%%%%%%%%%%%%%%%%%%%%%%%%%%%%%%%%%%%%%%%%%%%%%%%%%%%%%%%%%%%%%%%%%%%%%%%%%%%%%%%%
\section{TCP/IP}    
\label{OSI}              
%%%%%%%%%%%%%%%%%%%%%%%%%%%%%%%%%%%%%%%
\subsection{Referenčný model}        
Pozri \ref{reference_model}.

TCP/IP. Okrem vrstiev špecifikuje aj konkrétne protokoly. \emph{Internet} tu neznamená konkrétny Internet, ale medzi-sieť. Je špecializovaný na prenos binárnych dát. 
Daná vrstva zahŕňa protokoly s podobnými funkcionalitami a cieľmi. Každá vrstva využíva len funkcie susedných vrstiev. 
 
\begin{itemize}
\item Aplikačná. OSI + IMAP, POP, SOCKS, SSH.            
\item Transportná. Prenos dát medzi koncovými zariadeniami. Rovnaké ako OSI.     
\item Sieťová (Internet Layer). Adresácia a komunikácia medzi viacerými lokálnymi sieťami. IP (adresovanie), ICMP (reponse codes), IPsec.      
\item Linková. Kábeláž pre lokálne siete. MAC (6bytové adresy zariadení (sieťových kariet)). ARP (kto má MAC adresu s danou IP?), Ethernet, DSL, ISDN.      
\end{itemize}

Poznamenajme, že v skutočnosti je prvou hlavičkou hlavička linkovej vrstvy. 

Patrí sa porozprávať o jednotlivých protokoloch. 

%%%%%%%%%%%%%%%%%%%%%%%%%%%%%%%%%%%%%%%
\subsection{Porovnanie a kritika OSI a TCP/IP referenčných modelov}    

OSI - Open systems interconnection. IP - Internet\footnote{Nie ten konkrétny} Protocol suite. Oba modely delia komunikáciu na abstraktné vrstvy. 

IP je novší, špecializovaný na prenos digitálnych dát a používa sa prakticky vo všetkých počítačových sieťach (asi).
IP možnosť interpertovať ako inštanciu OSI.  

Kritika: niečo pohaluzím. 
        
%%%%%%%%%%%%%%%%%%%%%%%%%%%%%%%%%%%%%%%%%%%%%%%%%%%%%%%%%%%%%%%%%%%%%%%%%%%%%%%%
\section{Teoretické základy pre dátovú komunikáciu} 

Asi netreba vedieť, stačí tušiť. Ide o výsledky teórie informácie, ktorými sa zaoberá aj predmet \href{http://www.dcs.fmph.uniba.sk/texty/codebook.pdf}{Teória kódovania}. Ide o širokú tému. 

%%%%%%%%%%%%%%%%%%%%%%%%%%%%%%%%%%%%%%%
\subsection{Nyquistovo tvrdenie (Nyquist–Shannon sampling theorem)}             

\paragraph{Veta (Shannova verzia).} Nech $x(t)$ je vlnová funkcia\footnote{
Poskladaná z konečného počtu sínusov a kosínusov. Záujemcovia nech si prečítajú \href{http://en.wikipedia.org/wiki/Fourier_transform}{Fourier transform}.
} frekvencie v čase. Ak $(\forall t)x(t)\leq B$, tak je kompletne určená pomocou hodnôt vzdialených $\frac{1}{2B}$.

Ľudsky povedané, ak zachytávame vlnové signály, ktoré majú maximálnu frekvenciu $B$, tak nám stačí merať aktuálnu frekvenciu signálu každých $\frac{1}{2B}$. Naše merania potom jednoznačne určujú vlnovú funkciu $x(t)$.

V praxi vieme namerať diskrétne hodnoty spojitého signálu, ktoré podľa spomínanej vety vieme rekonštruovať naspäť na spojitú funkciu (niečo ako interpolačné polynómy - pre tých, čo mali numeriku). 
  
%%%%%%%%%%%%%%%%%%%%%%%%%%%%%%%%%%%%%%%
\subsection{Shannonove odhady a ich dôsledky}  
Odhady mám pass. 
Tipol by som, že frekvencie signálov v sieťových spojeniach sú nastavené tak, aby sa stíhali čítať. Alebo z druhej strany, cielená prenosová rýchlosť určuje frekvenciu priameho spojenia. 
Používa sa napr. aj v digitálnych fotoaparátoch. 


%%%%%%%%%%%%%%%%%%%%%%%%%%%%%%%%%%%%%%%%%%%%%%%%%%%%%%%%%%%%%%%%%%%%%%%%%%%%%%%%
\section{Dátové prenosy}     
\begin{itemize}
\item \emph{Sériová komunikácia}. Po komunikačnom kanály sa naraz posiela iba jeden bit informácie. Jeden kábel, jeden bit. Výhodou je jednoduchosť a cena. 
\item \emph{Paralelná komunikácia}. 
Po komunikačnom kanály sa naraz posiela viacero bitov informácie. 
Kanál pozostáva z viacerych prenosových médií (káblov). 
Nevýhodou sú problémy so synchronizáciou, napr. kvôli zachovania poradia dát je nutné čakať na všetky bity v jednej dávke (\href{http://en.wikipedia.org/wiki/Clock_skew}{Clock skew}) 
alebo \href{http://en.wikipedia.org/wiki/Crosstalk}{Crosstalak}, keď signál v jednom kábli ovplyvňuje signály v okolitých kábloch (nedostatočné tienenie). 
Výhodou je teoreticky vyššia prenosová rýchlosť. 
\end{itemize}

V minulosti sa sériová komunikácia používala hlavne na veľké vzdialenosti (cost-effective oproti paralelej) 
a paralelná na menšie (high-throughput). 
Sériové sa kvôli jednoduchosti (a dnešným prenosovým rýchlostiam) používajú aj na krátke vzdialenosti, napr serial bus (zbernica), alebo PCI (paralel) na PCI Express (serial). 
Sériové spojenia je možné pretaktovať na rádovo vyššie rýchlosti ako paralelné.  

%%%%%%%%%%%%%%%%%%%%%%%%%%%%%%%%%%%%%%%
\subsection{UART (Universal asynchronous receiver/transmitter)} 

Konverzia medzi sériovým a paralelným prenosom.
Asynchrónne sériovo - každý bit je prenášaný v 8bitovom bloku, ktoré sú rozdelené start (nasleduje 8 butov) a stop (poslaných 8 bitov) signálmi.
Dáta zo sériových prenosov sa buffrujú, aby sa potom mohli poslať paralelne. 
Používa sa v integrovaných obvodoch, materských doskách, sériových portoch, modemoch... .       
       
%%%%%%%%%%%%%%%%%%%%%%%%%%%%%%%%%%%%%%%
\subsection{USRT (Universal Synchronous Receiver-Transmitter)}   

Nepotrebuje start / stop signály, hodiny (clock) sú synchronizované rovnakou časovou jednotkou.
UART podporujúci USRT sa nazýva USART (universal synchronous/asynchronous receiver/transmitter).
Moderné UART majú podporu pre synchrónny mód. 
  
%%%%%%%%%%%%%%%%%%%%%%%%%%%%%%%%%%%%%%%
\subsection{Synchronizácia}   
In synchronous transmission, the clock data is recovered separately from the data stream and no start/stop bits are used. This improves the efficiency of transmission on suitable channels since more of the bits sent are usable data and not character framing. An asynchronous transmission sends no characters over the interconnection when the transmitting device has nothing to send; but a synchronous interface must send "pad" characters to maintain synchronization between the receiver and transmitter. The usual filler is the ASCII "SYN" character. This may be done automatically by the transmitting device.

%%%%%%%%%%%%%%%%%%%%%%%%%%%%%%%%%%%%%%%%%%%%%%%%%%%%%%%%%%%%%%%%%%%%%%%%%%%%%%%%
\section{Elektromagnetické spektrum a bezdrôtové prenosy}  

Mám pass. 

%%%%%%%%%%%%%%%%%%%%%%%%%%%%%%%%%%%%%%%%%%%%%%%%%%%%%%%%%%%%%%%%%%%%%%%%%%%%%%%%
\section{Prenosové média a sieťové komponenty} 
Mám dosť pass, nie som predsa fiitkár. 

%%%%%%%%%%%%%%%%%%%%%%%%%%%%%%%%%%%%%%%
\subsection{Techniky prepojovania sietí}  
Asi rôzne topológie. 

%%%%%%%%%%%%%%%%%%%%%%%%%%%%%%%%%%%%%%%
\subsection{Charakter kabeláže}     
Sivá, biela, čierna. 

%%%%%%%%%%%%%%%%%%%%%%%%%%%%%%%%%%%%%%%
\subsection{Štruktúrovaná kabeláž}  
\paragraph{Koaxiálny kábel.}
\begin{center}
\includegraphics{coaxial.jpg}
\end{center}
\begin{itemize}
\item A - Vonkajší plastový obal. 
\item B - Medené tienidlo (pred vesmírnym šumom). 
\item C - Elektrický izolátor.
\item D - Medené jadro - prenosové médium. 
\end{itemize}

Je ich asi 10 typov. 

%%%%%%%%%%%%%%%%%%%%%%%%%%%%%%%%%%%%%%%
\subsection{Sieťové zariadenia}           

\begin{itemize}
\item Gateway - konverzia dát medzi rôznymi sieťovými protokolmi, prístupový / výstupový bod siete. Používa vrstvy 4-7 OSI modelu. 
\item Hub - fyzicky je centrom hviezdicovej fyzickej topológie (ethernetové káble). Hub preposiela správu od každého každému a vytvára tak jednu kolíznu doménu. Logickú topológiu má teda zbernicovú. Druhá vrstva OSI. 
\item Bridge - rámce s adresou (MAC adresou\footnote{
Bond, James Bond. 
}) nepreposiela každému. Ak adresu nepozná, tak sa spýta všetkých, koho adresa je (niečo ako ARP). Ďalej si už adresu pamätá (ak sa dlho nepoužila, môže ju zmazať). Druhá vrstva OSI. 
\item Switch - konceptuálne rovnaké ako bridge. Rozdiel je v tom, že switch skúma iba hlavičku prechádzajúcich rámcov, bridge kontroluje aj integritu (checksum). Druhá vrstva OSI. \href{http://en.wikipedia.org/wiki/Ethernet#Bridging_and_switching}{Rozdiely medzi bridgeom a switchom}. 
\item Router - pozri \ref{routing}. 
\item Repeater - zosilňuje signály a porušené signály znovu posiela. Prvá vrstva OSI. 
\item Proxy server - prostredník komunikácie (man in the middle). Cieľom je uľahčiť komunikáciu pre odosielateľa správy. Najvyššia vrstva.  
\label{firewall} 
\item Firewall - monitoruje hlavičky paketov na sieťovej vrstve a na základe ich adries a portov sa rozhoduje, či komunikáciu prepošle ďalej, blokuje (a pošle dôvod prečo, napr. ICMP) alebo zahodí (a má pass zdôvodnenie). 

\emph{Bezstavový} firewall má len takéto pravidlá. \emph{Stavový} firewall si vie pamätať vytvorené virtuálne spojenia (ako napr. TCP). To je nutné v prípade, že klient vytvára virtuálny okruh a očakáva odpoveď, ktorú by bezstavový firewall odmietol (stavový si pamätá vytvorený virtuálny okruh a dovolí spätnú správu). Stavový firewall sa používa čo najbližšie ku koncovému užívateľovi (keď počet TCP spojení môže byť obrovský a pamäť je konečná - je nutné premazávať). 

Blacklist - niektorým zakáže, whitelist - len niektorým povolí. 

\label{NAT}
\item Network address translsator (NAT) - pozri \href{http://netlab.dcs.fmph.uniba.sk/siete/cviko8/}{praktiká zo sietí}. Prepisuje adresy a porty v hlavičkách paketov. Účelom môže byť skrytie privátnych adries\footnote{
Privátne adresy sa nemôžu vyskytnúť na Internete (konkrétny). 
} za jednu verejnú (vytvoreným TCP spojeniam je možné pamätať si koncové adresy a porty). Takýto nat sa nazýva aj SNAT (source NAT). Existuje aj DNAT (destination NAT), ktorý pomocou \href{http://en.wikipedia.org/wiki/Port_forwarding}{port-forwardingu} sprístupňuje privátne služby verejnosti (služba publikuje verejnú IP adresu a špeciálny port, povie to DNATu a ten mu preposiela komunikáciu zo zvoleného portu). Ak protokol nevytvára virtuálne okruhy, tak SNAT nevie pracovať stopercentne (keďže nevie, komu má preposielať spätné správy).  
\item Sieťová karta - každý vie. 
\item Modem (modulátor-demodulátor) - transformuje analógový signál na digitálny a naopak. V prípade Internetu sa využíval na pripojenie pomocou telefónnych liniek (dial-up).
\end{itemize}
 
%%%%%%%%%%%%%%%%%%%%%%%%%%%%%%%%%%%%%%%%%%%%%%%%%%%%%%%%%%%%%%%%%%%%%%%%%%%%%%%%
\section{Linková vrstva (IP)}        
%%%%%%%%%%%%%%%%%%%%%%%%%%%%%%%%%%%%%%%
\subsection{MAC adresa (Media Access Control address)}     
Unikátny identifikátor pre sieťové zariadenia (sieťové karty napr.),
dodávané ich výrobcom. Má 6 byteov, prvé tri pre výrobcu (OUI - Organizationally Unique Identifier), ďalšie tri pre unikátny identifikátor (NIC - Network interface controller). Špeciálne $FF:FF:FF:FF:FF:FF$ je broadcastová adresa. 
          
MAC -  Media Access Control - akým spôsobom sa pristupuje k zdielanému médiu (na ktorom môže vysielať viacero staníc naraz).          
%%%%%%%%%%%%%%%%%%%%%%%%%%%%%%%%%%%%%%%
\subsection{IEEE štandardy 802 pre LAN (a WAN)}  
Február 1980. Rôzne veľkosti paketov. Rozdeľuje dátovú vrstvu na MAC  (predošlý odstavec) a LLC - Logical Link Control, ktorý umožňuje viacerým rôznym protokolom "paralelný" prístup k médiu (komunikačný kanál).

Je ich viacero, napr.:
\begin{itemize}
\item 802.3 Ethernet      
\item 802.11 Wireless LAN
\item 802.15.1 Bluetooth 
\end{itemize}

%%%%%%%%%%%%%%%%%%%%%%%%%%%%%%%%%%%%%%%
\subsection{FDDI (Fiber Distributed Data Interface)}  
Sieťová karta. Umožňuje prenos 100Mbit/s, na kružnicovej topológií (zdielané médium). Používal sa v LAN. Nahradený Fast Ethernet.    
                 
%%%%%%%%%%%%%%%%%%%%%%%%%%%%%%%%%%%%%%%
\subsection{Fast Ethernet}            
Lacný, všade používaný (nazývaný aj LAN kábel).
100Mbit / s. Hviezdicová topológia (centrom je Hub, Switch alebo Router).
Používa CSMA/CD (Detekcia kolízií s náhodným čakaním (podľa exponenciálnej distribúcie pravdepodobnosti = \emph{exponential backoff}) v prípade kolízie). Viacero verzií podľa druhu vodiča (medené / optické).

%%%%%%%%%%%%%%%%%%%%%%%%%%%%%%%%%%%%%%%
\subsection{Gigabit Ethernet}       
Od roku 2010 "common and economical" (štandardizované 1998). Pomocou switchov je umožnená full-duplex komunikácia (obojsmerná). 
Zas asi 10 verzií (od 25m po 70 km), (asi vždy) optické vlákna. 

%%%%%%%%%%%%%%%%%%%%%%%%%%%%%%%%%%%%%%%
\subsection{10 Gigabit Ethernet}  
Len full-duplex (obojsmerne), nedá sa použiť s hubmi ani s CSMA/CD. 
Od roku 2002. Problémom sú 10Gbit routre (podľa istých zdrojov realizovateľné na moderných grafickách kartách, lepšie ako bitcoiny). Optické káble. Zas kopa verzií, typovo rovnakých. 

%%%%%%%%%%%%%%%%%%%%%%%%%%%%%%%%%%%%%%%
\subsection{802.11 (Wireless LAN)}  

Pozri \href{http://netlab.dcs.fmph.uniba.sk/siete/cviko2/}{Cviká z počítačových sietí}, ďalej len výcuc. 
CSMA (Carrier Sense Multiple Access - Collision avoidance) - ak je linka obsadená, tak exponential backoff. Zdielané médium - 2.4GHz až 5GHz pásma. 
Antény ovplyvňujú kvalitu. Viacero protokolov (rôzne prenosové rýchlosti).

Stanica (Station) je zariadením vo WiFi sieti. BSS (Basic Service Set) - zariadenia, ktoré vedia medzi sebou komunikovať v dvoch režimoch:
\begin{itemize}
\item \emph{Ad-hoc} režim, alebo tiež IBSS (Independent Basic Service Set) sa používa v prípade, keď je treba jednoduchým spôsobom vytvoriť bezdrôtovú sieť. Prvá stanica vytvárajúca sieť náhodne zvolí BSSID, začne vysielať majáčiky (beacons) oznamujúce prítomnosť siete a ostatné stanice komunikujú priamo medzi sebou. Ak stanice dlhšiu dobu nedostanú beacon, predpokladajú, že prvá stanica vypadla a zvyšné stanice v náhodnom čase začnú vysielať beacony (prvá stanica vyhráva).
\item \emph{Infrastructure} režim obsahuje stanicu, ktorá zároveň plní úlohu prístupového bodu -- AP (Access Point), s ktorou komunikujú všetky ostatné stanice (dve stanice môžu vždy komunikovať iba cez AP).
\end{itemize}
     
Problém s bezpečnosťou (ľahké zachytávania paketov). Zabezpečenie: 
\begin{itemize}                                                 
\item Bez šifrovania -- najstarší, najjednoduchší, najnebezpečnejší.
\item WEP (Wired Equivalent Privacy). Šifra RC4 (64-128bitový)        
\item WPA (Wi-Fi Protected Access). Namiesto RC4 TKIP (Temporal Key Interchange Protocol) - časté menenie kľúčov (neprichádza k celkovému odhaleniu komunikácie pri odhlení jedného kľúča).
\item WPA2. Novšia verzia. 
\item EAP (použivaný aj v point-to-point). Má oveľa širšie možnosti autentifikácie (meno/heslo, certifikát, ...).
\end{itemize}
      
Distribučný systém - prepojenie viacerých WiFi routrov (staníc) medzi sebou Ethernetom alebo bezdrátovo. Treba myslieť na MAC adresy, keďže nemusia byť pod daným routrom. 

Mesh networking (také peer-to-peer). Každý je rovnocenný, preposiela pakety ďalej (chová sa ako wifi router). Nie je rozšírené. 
      
%%%%%%%%%%%%%%%%%%%%%%%%%%%%%%%%%%%%%%%
\subsection{802.15 (Wireless PAN)} 
PAN - personal area network - len medzi zariadenia používané jednou osobou (napr.:mobil, tablet, notebook, desktop, mp3 player) - nazýva sa aj \emph{Body Area Network}.
Bluetooth, WiFi (staré infraporty a ďalšie). 
%%%%%%%%%%%%%%%%%%%%%%%%%%%%%%%%%%%%%%%
\subsection{802.16 (Wireless broadband)}
Ako mobilná sieť, ale ponúka bezdrátové pripojenie k Internetu.
Viac na \href{http://en.wikipedia.org/wiki/Broadband\_Wireless\_Access}{Broadband Wireless Access}. 
       
%%%%%%%%%%%%%%%%%%%%%%%%%%%%%%%%%%%%%%%
\subsection{Bluetooth}           
Výmena dát na krátku vzdialenosť. Každý pozná. 
Dosť komplikovaná implementácia. 
                                                 
%%%%%%%%%%%%%%%%%%%%%%%%%%%%%%%%%%%%%%%%%%%%%%%%%%%%%%%%%%%%%%%%%%%%%%%%%%%%%%%%
\section{Sieťová vrstva (OSI)}     
Dôležitá otázka. Dobre popísané v \href{http://netlab.dcs.fmph.uniba.sk/siete/cviko3/}{praktikách zo sietí}. Je ekvivalentná internet layer v IP. 
Cieľom vrstvy je, aby vedeli počítače medzi sebou komunikovať bez ohľadu na použitie konkrétnej linkovej vrstvy (ethernet, WiFi, dial-up, ISDN, ADSL, ...). 
                 
Odporúčam pozrieť si niečo o \href{http://en.wikipedia.org/wiki/Ipsec}{IPsec} (AH a ESP).                  
                 
%%%%%%%%%%%%%%%%%%%%%%%%%%%%%%%%%%%%%%%
\subsection{Interná organizácia sieťovej vrstvy} 

Hlavnými cieľmi sú: 
\begin{itemize}
\item Nevytvárať virtuálne okruhy. Zaručuje iba end-to-end komunikáciu. 
\item Adresný priestor. Každý užívateľ siete má unikátnu adresu (nemusí byť pravda v prípade použitia NATu).  
\item Smerovanie správy. Nie každý vie priamo posielať správy hocikomu po linkovej vrstve, keďže nemusia byť spojení káblom. Preto sa správa musí posielať cez viacero vrstiev - smerovanie (niečo ako GPS). 
\end{itemize}

%%%%%%%%%%%%%%%%%%%%%%%%%%%%%%%%%%%%%%%
\subsection{Príklad prenosu packetov cez sieť} 

Routre s routovacími tabuľkami. Dve prepojené siete. 

%%%%%%%%%%%%%%%%%%%%%%%%%%%%%%%%%%%%%%%
\subsection{Uzly a smerovacie tabuľky}     
Uzly ???. Asi sa myslia sieťové zariadenia, ktoré sú spojené s viacerymi sieťovými zariadeniami (hub, switch, router, NAT, ...). 
Smerovacie tabuľky v \ref{routing_tables}. 

%%%%%%%%%%%%%%%%%%%%%%%%%%%%%%%%%%%%%%%
\subsection{IP protokol}    
Suverénne najpoužívanejší protokol. Aktuálne sa prechádza na IPv6 (128bitové adresy), kvôli vyčerpaniu adresného priestoru IPv4 (32bitové adresy). 

\paragraph{Fragmentácia.}
IP definuje pohyblivú veľkosť packetu (po $2^16$). Použité linkové vrstvy môže mať rôzne obmedzenia na veľkosť paketu. Preto IPv4\footnote{
IPv6 fragmentáciu zakazuje, je nutné dopredu zistiť MTU (resp. použiť minimálnu). 
} umožňuje packety v uzloch \emph{fragmentovať} - rozdeliť. Ďalší uzol potom tieto pakety spojí naspäť. \emph{MTU} - Maximum transport unit - označuje maximálnu veľkosť paketu prenositeľného cez vytvorený virtuálny okruh (alebo spojenie). 

\paragraph{Adresácia.}
IPv4 adresy sú 4 bajtové, zapisujú sa $123.45.67.255$. Spolu s ňou sa posiela \emph{sieťová maska} (subnet mask), ktorá rozdeľuje adresu na \emph{adresu siete} (network identifier) a \emph{adresu zariadenia} (host identifier). Maska je niekoľko jednotiek nasledovaných nulami (32 možností), píše sa ako číslo udávajce počet jednotiek za IP adresu: $123.45.67.255/24$. 

IP vyhradzuje špeciálne adresy. 127.0.0.1 je loopback (localhost) a pre lokálne použitie $10.0.0.0/8$, $172.16.0.0/12$ a $192.168.0.0/16$. Broadcastová adresa vznikne ako adresa siete vyplnená jednotkami: $123.0.0.1/8 \mapsto 123.255.255.255$ alebo 
$158.195.213.69/8 \mapsto 158.195.213.255$.
Adresy môžu byť staticky nastavené (administrátorom), alebo získané dynamicky (pomocou DHCP, pozri \ref{DHCP}) (dá sa určite nastaviť vo vašom obľúbenom operačnom systéme). 

\paragraph{ARP.}
IP adresy musia byť v konečnom dôsledku preložené na MAC adresy. Na tento účel slúži protokol \emph{ARP - \href{http://en.wikipedia.org/wiki/Address\_Resolution\_Protocol}{Address Resolution Protocol}}. Prebieha v dvoch fázach. Najprv uzol pošle broadcast všetkým na sieti s otázkou "Who has IPadresa?". Dotyčné zariadenie odpovedá svojou MAC adresou. Uzol si zobrazenie $IPaddress \mapsto MACaddress$ môže istý čas pamätať.  

\paragraph{ICMP (Internet Control Message Protocol).}
Kontrolná správa. Prenáša len jeden flag - status. Napr. firewall môže poslať ICMP v prípade, že správu blokoval. 

\paragraph{Time to live.}
Môže sa stať, že routovacie tabuľky môžu tvoriť cykli (resp. nevedia doručiť daný paket).
Preto IP definuje TTL položku, ktorá určuje maximálny počet skokov, ktorý môže paket v sieti urobiť. 
           
%%%%%%%%%%%%%%%%%%%%%%%%%%%%%%%%%%%%%%%
\subsection{Smerovanie}         
\label{routing}

Ide o to doručiť správy v prípade, že medzi komunikujúcimi zariadeniami neexistuje priame spojenie. 

\paragraph{Spôsoby doručenia.}
\begin{itemize}
\item Unicast - správa je určená jednému konkrétnemu.         
\item Broadcast - správa je určená všetkým v sieti (neposiela sa ďalej medzi siete - internetwork).
\item Multicat - správa sa posiela množine vrcholov. 
\item Anycast - správa sa doručuje práve jednému z množiny vrcholov (najčastejšie najbližšiemu).
\end{itemize}                               
                
\paragraph{Používané zariadenia.}
\begin{itemize}
\item Router. Pomocou \emph{routovacej tabuľky} $adresa_prijemcu \mapsto preposlat_na$ smeruje správy ďalej, resp. určenému príjemcovi.         
\item Bridge. Príjemcu hľadá pomocou floodingu - opýta sa každého v sieti (napr. protokol ARP \ref{ARP}, resp Neighbor Discovery v IPv6). Keď našiel, tak si ho zapamätá. Používa sa v malých sieťach, ako najnižší router.  
\item Gateway (nazývané aj protocol converters). Ich hlavným účelom je konverzia dát medzi dvoma protokolmi, resp. sú vstupnými bránami do iných \{$\epsilon$,pod,nad\}sietí. Využívajú sa na ľubovoľnej vrstve. 
\item Firewall. Preposiela, Zahadzuje (ničí) alebo Blokuje (pošle echo prečo) správy na základe jednoduchých pravidiel (port, protokol, IP adresa). Pracuje na sieťovej vrstve. Viac o firewalloch v \ref{firewall}. 
\item Switch. Narozdiel od hubu (kotrý preposiela správy všetkým ostatným), posiela správu len jej príjemcovi. 
\end{itemize}  

\label{routing_tables}
\paragraph{Routovacie tabuľky.} 
\begin{itemize}
\item Statické. Administrátor siete natvrdo zadá, kadiaľ sa majú preposielať dáta určené pre danú adresu.           
\item Adaptívne. Vo veľkých dynamických sieťach (napr. Internet) nie je možné konfigurovať staticky, preto si routre medzi sebou pravidelne vymieňajú informácie (každý router pozná zopár iných, napr. štandardný domáci router pozná router ISP). 
\end{itemize} 

Algoritmy na vypĺňanie routovacích tabuliek:
\begin{itemize}
\item Distance vector algorithms - Bellman-Ford ($O(n^3)$). Každej hrane sa priradí cena, potom zbehne algoritmus na výpočet vzdialeností (násobenie matice susednosti), správy od routra $R$ pre $A$ sa smerujú po najkratšej ceste. Tento prístup sa využíva v malých (lokálnych) sieťach. Môže sa použiť aj zložitejší Dijsktrou algoritmus, ale pre niektoré zariadenia môže byť príliš zložiťý.           
\item OLSR - Optimised Link State Routing - prehľadáva sa len do hĺbky 2, používa sa v ad-hoc sieťach (netreba vedieť). 
\item Path vector protocol - vytvorí sa hierarchická štruktúra, každý vrchol je zodpovedný za svojich priamych potomkov, použije sa hierarchická adresácia. Táto hierarchická štruktúra je updateovaná pomocou špecializovaných protokolov (\href{http://tools.ietf.org/html/rfc1322}{RFC-1322}). Používa s v medzisieťovom smerovaní.  
\end{itemize}
    
%%%%%%%%%%%%%%%%%%%%%%%%%%%%%%%%%%%%%%%
\subsection{CIDR (Classless Inter-Domain Routing)} 
Do roku 1993 boli všetky adresy rozdelené len do piatich tried a susediace adresy nemuseli susediť aj topologicky. Od roku 1993 CIDR rozdeľuje adresy na adresu siete a adresu zariadenia (maska siete, spomínané vyššie). 

Dosahuje sa tak jednoduchší routing (stačí poslať správu sieti) a rozširuje sa potenciálny počet IPv4 adries (koncové zariadenia nemusia mať unikátnu adresu, NAT). 

%%%%%%%%%%%%%%%%%%%%%%%%%%%%%%%%%%%%%%%
\subsection{Tunelovanie}          
Pozri \href{http://netlab.dcs.fmph.uniba.sk/siete/cviko11/}{cviká zo sietí}.
                                  
Účelom tunelu je vytvoriť virtuálne spojenie, cez ktoré sa dá komunikovať tak, akokoby boli koncové zariadenia spojené priamo. 
Prakticky sa paket $P$ (hocijakej vrstvy) zabalí do špeciálneho tunelovacieho paketu - obalu (sieťová vrstva). Obal sa medzi sieťami prenáša, prechádza cez routre, firewally, NAT, proxy, ... (ktorým sa $P$ takýmto spôsobom úspešne vyhýba). 
Ak dosiahne router v koncovej sieti, tak sa rozbalí a $P$ sa pošle, akokeby vznikol na tej koncovej sieti. 

Výhody tunelovania sú, že je možné celý paket poslať bezpečne po sieti. Je možné zabezpečiť autenticitu (kto paket posielal), hlavičku a obsah (šifrovaním) a integritu (checksum). 
To v bežných paketoch nie je možné v takej miere, lebo niektoré časti sa pri prechode cez routre menia (TTL, header checksum,...).  

\emph{VPN (Virtual private network)} pozri \href{http://en.wikipedia.org/wiki/Virtual_private_network}{Wikipédiu}.
Tunelovaním sa vytvára virtuálna lokálna sieť, ktorá spája množinu klientov. Títo klienti posielajú potom medzi sebou správy, akokeby boli na izolovanej fyzickej sieti (akékoľvek správy, nie je možné odpočúvať, vieme zaručiť kto správu poslal a či sa zmenila). 
VPN sú populárne, nemôže chýbať v každej modernej firme. 
Jednou z implementácií je OpenVPN. 
                                                           
%%%%%%%%%%%%%%%%%%%%%%%%%%%%%%%%%%%%%%%%%%%%%%%%%%%%%%%%%%%%%%%%%%%%%%%%%%%%%%%%
\section{Transportná vrstva (IP, OSI)}       
Dôležitá otázka. Pozri \href{http://netlab.dcs.fmph.uniba.sk/siete/cviko4/}{praktiká z počítačových sietí}. 
   
Cieľom transportnej vrstvy je zabezpečiť transparentný prenos dát medzi ľubovoľnými dvoma používateľmi (procesmi) v sieti. Keďže na jednom počítači môže bežať viac procesov, úlohou transportnej vrstvy je aj umožniť paralelný prenos viacerých prúdov dát. Protokoly to riešia tak, že rozšíria sieťové adresy o ďalší prvok - čislo portu. Adresa na transportnej vrstve je potom tvaru (sieťová adresa, číslo portu) a jednoznačne identifikuje tzv. socket cez ktorý komunikuje nejaký proces.
   
%%%%%%%%%%%%%%%%%%%%%%%%%%%%%%%%%%%%%%%
\subsection{UDP (User Datagram Protocol)}   
Najjednoduchší. Nezabezpečuje pratkicky nič, len sa snaží správu doručiť (aspoň raz). 
Rozširuje IP hlavičku iba o čísla portov a checksum. 
Používa sa, keď nie je spoľahlivosť doručenia požiadavkou. Napr. streamovanie porna, keď nám až tak nevadí, keď sa niektoré pakety stratia. 
Používa sa aj pri tunelovaní. 

%%%%%%%%%%%%%%%%%%%%%%%%%%%%%%%%%%%%%%%
\subsection{TCP (Transmission Control Protocol)}   
Pozri \href{http://netlab.dcs.fmph.uniba.sk/siete/cviko5/}{praktiká z počítačových sietí} \footnote{Väčšina textu tejto sekcie (TCP) je skopírovaná z danej stránky, dúfame, že sa tým neporušili žiadne copyrighty.}.

Transmission control protocol (TCP) je najpoužívanejším protokolom transportnej vrstvy na Internete. Služby ako World Wide Web (HTTP protokol), e-mail (protokoly SMTP, POP3, IMAP) a mnohé iné používajú TCP na prenos údajov.
Vytvára virtuálne okruhy pre prenos prúdu dát a zaručuje spoľahlivosť prenosu (preposielanie stratených a preusporiadanie doručených). Príjemca potvrdzuje príjem.  

\paragraph{Hlavička}
\begin{center}
\begin{tabular}{ r | l }       
  Offset (bajt) & Popis\\
\hline              
  0-1 & Číslo zdrojového portu (Source port)\\
  2-3 & Číslo cieľového portu (Destination port)\\
  4-7 & Poradové číslo (Sequence number)\\
  8-11 & Potvrdzujúce číslo (Acknowledgment number (if ACK set))\\
  12-13 & Príznaky\\
  14-15 & Veľkosť okna (Window Size)\\
  16-17 & Kontrolný súčet (Checksum)\\
  18-19 & Smerník na urgentné dáta (nepoužívané) (Urgent pointer (if URG set))\\
  20-... & Voliteľné rozšírenia (Options)\\
\end{tabular}
\end{center}

Poradové číslo určuje do ktorej časti prúdu patria dáta, ktoré sa prenášajú v danom segmente. Na základe tohto poľa vie príjemca poskladať segmenty prijaté v nesprávnom poradí, prípadne zistiť, že niektoré dáta neprijal resp. prijal viackrát. Na začiatku spojenia sa toto pole určí ľubovoľne (nezávisle pre každý smer komunikácie) a následne sa zväčšuje s každým preneseným bajtom.

\begin{center}
\includegraphics[scale=0.5]{tcp_window.png}
\end{center}

Príjemca oboznamuje odosielateľa, že úspešne prijal dáta tým, že vyšle vlastný TCP segment v ktorom nastaví potvrdzujúce číslo na hodnotu o jeden väčšiu ako poradové číslo posledného prijatého bajtu (t.j. potvrdzujúce číslo má hodnotu poradového čísla, ktorú príjemca očakáva ako nasledovnú).

\paragraph{Vytváranie, riadenie a ukončovanie spojenia.} 

\begin{center}
\begin{tabular}{ r | l }       
  Bit & Popis\\
\hline              
  0-3 & Smerník na dáta\\              
  4-7 & Rezervované\\              
  8 & Congestion Window Reduced (CWR)\\              
  9 & ECN-Echo (ECE)\\              
  10 & Smerník na urgentné dáta je platný (URG)\\              
  11 & Potvrdzujúce číslo je platné (ACK)\\              
  12 & Push (PSH)\\           
  13 & Resetovanie spojenia (RST)\\           
  14 & Synchronizovanie poradových čísel (SYN)\\           
  15 & Žiadne ďaľšie údaje od odosielateľa (FIN)\\
\end{tabular}
\end{center}

PSH - vykonaj ihneď a nebufruj správy na príjemcovej strane. RST - ukončenie spojenia. 
SYN - inicializuje TCP spojenie (three way handshake) a SYNchronizuje poradové čísla,
príjemca odpovedá SYN a ACK. Spojenie sa ukončuje FIN. 

Nečaká sa na potvrdenie (ACK) každého paketu (veľmi veľmi pomalé), posiela sa aj viacero paketov naraz (sliding window). 

\paragraph{Kontrola zahltenia (Congestion control)}

TCP obsahuje viacero mechanizmov, ktorými sa snaží zabrániť zahlteniu siete. Každá strana si udržuje tzv. "congestion window", čo je odhad maximálneho množstva dát, ktoré sa môžu poslať súčasne (bez čakania na potvrdenie) bez výpadkov.

Okno sa na začiatku exponenciálne zväčšuje, kým nepríde k výpadku, vtedy sa resetuje. 

%%%%%%%%%%%%%%%%%%%%%%%%%%%%%%%%%%%%%%%
\subsection{SSL/TLS}                 
Šifrovanie určené pre transportnú vrstvu (SSL je staršia verzia). 
Asymetrické šifrovanie pre výmenu kľúčov (štandardne Diffie-Hellman), 
symetrické šifrovanie pre nečitateľnosť a
MAC na integritu a autenticitu\footnote{
Istota, že správu vytvoril naozaj užívateľ s danou IP adresou. 
} (napr. podpisovaním hashu správy). 
Rozširuje štandardné protokoly ako HTTP, IMAP, ... na HTTPS, IMAPS, ... .

Zjednodušený priebeh protokolu:
\begin{itemize}
\item Odosielateľ (klient) a príjemca (server) si vymenia informácie o nastaveniach. 
\item Klient overí autenticitu servra (môže aj servert klienta). Ak nemá zapamätaný jeho verejný kľúč, tak konzultuje certifikačné autority. 
\item Klient sa dohodne so serverom na spoločnom tajomstve pre túto komunikáciu (pomocou DH protokolu). Šifruje verejným kľúčom servra. 
\item Zo spoločného tajomstva sa vygenerujú session keys, ktorými sa správy ďalej šifrujú. 
\item Potvrdí sa session key. 
\item Môže sa bezpečne komunikovať.
\end{itemize} 

\paragraph{Certifikačné autority.}
Aby klient vedel overiť autenticitu, musí veriť spojeniu verejný kľúč - IPadresa. 
Viera (trust) týchto dvojíc je outsourcovaná na \emph{certifikačné} autority,
ktoré podpisujú takéto certifikáty. Certifikačné autority tvoria stromovú štruktúru. Koreňu stromu sa implicitne verí. 
Certifikát, ktorý nie je registrovaný CA je neplatný a všetky prehliadače na to upzorňujú. 
            
%%%%%%%%%%%%%%%%%%%%%%%%%%%%%%%%%%%%%%%%%%%%%%%%%%%%%%%%%%%%%%%%%%%%%%%%%%%%%%%%
\section{Aplikačná vrstva a podporné protokoly}     
Dôležitá otázka.   


%%%%%%%%%%%%%%%%%%%%%%%%%%%%%%%%%%%%%%%
\subsection{DHCP (Dynamic Host Configuration Protocol)} 
Pozri \href{http://netlab.dcs.fmph.uniba.sk/siete/cviko4/}{praktiká z počítačových sietí}. 
  
Protokol slúži na získanie IP adresy v lokálnej sieti. 
Pri úvodnom pripojení klient nemá IP adresu a potrebuje nejakú získať. Kontaktuje preto DHCP server UDP paketom na adresu 255.255.255.255 (a port 67) so zdrojovou adresou 0.0.0.0\footnote{
Je to trocha hack, posielať IP pakety, keď ešte IP stack nebol inicializovaný.
} s obsahom DHCPDISCOVER, v ktorej uvádza svoju MAC adresu a zoznam nastavení. 

Vo všeobecnosti server odpovedá broadcastom\footnote{
Keďže klient ešte nemá nakonfigurovaný IP stack. 
} DHCPOFFER (na port 68) a navrhuje aspoň jednu IP adresu, z ktorej si klient vyberie a oznámi broadcastom DHCPREQUEST a nastaví si IP stack. Server odpovie unicastom DHCPACK.
  
%%%%%%%%%%%%%%%%%%%%%%%%%%%%%%%%%%%%%%%
\subsection{DNS (Domain Name System)}   
Pozri \href{http://netlab.dcs.fmph.uniba.sk/siete/cviko4/}{praktiká z počítačových sietí}. 

Distribuovaná databáza bijektívneho zobrazenia znakových reťazcov (dómen) a IP adries. Jedny sa ľahko pamätajú ľudom, druhé počítačom. Priestor doménových mien je strom, pričom každý vrchol toho stromu môže mať priradené rôzne typy záznamy. Koreň stromu sa označuje bodkou. Každému vrcholu stromu (okrem koreňa) je priradené meno pozostávajúce z písmen, číslic a pomlčky.

DNS servre (nameservre) tvoria hierarchickú stromovú štruktúru. Každý z nameservrov má na starosti svoju zónu. Pre každú zónu existuje aspoň jeden hlavný (master) NS a podriadený (slave) si záznamy kopírujú (kvôli distribuovanému prístupu). 

Položky DNS záznamu: 
\begin{itemize}
\item Meno. Plné meno vrcholu ku ktorému sa viaže daný záznam, získané zreťazením mien všetkých vrcholov na ceste ku koreňu, pričom jednotlivé mená oddeľujeme bodkou.
\item Trieda. Najzaujímavejšia trieda je IN (internet), ešte sú definované aj triedy CN (chaos) a HS (hesiod). 
\item Typ:
\begin{itemize}
\item A,AAAA - autoritatívne záznamy. Tento nameserver je za ne zodpovedný. 
\item PTR - opačný preklad (ip na name). 
\item NS - delegovanie autority na iný nameserver (presmerovanie). 
\item MX - mail exchange - mailový server. 
\item SOA - start of authority - popisuje údaje o samotnej zóne. 
\end{itemize}
\item TTL - dokedy je záznam platný a môže sa cacheovať. 
\item Ďalšie dáta. 
\end{itemize}

Dva režimy pracovania DNS servra: 
\begin{itemize}
\item Nerekurzívny. DNS server vracia len autoritatívne záznamy (A), ak nemá záznam, tak vráti čiastočný výsledok: NS záznam ukazujúci na nameserver, ktorý má viac informácií (klient sa potom rekurzívne pýta). 
\item Rekurzívny. DNS server sa sám rekurzívne pýta ostatných nameserverov na odpoveď.
\end{itemize}

Odpovede sa môžu cacheovať, maximálne ale TTL čas. DNS je na porte 53, pristupuje sa UDP alebo TCP. 

%%%%%%%%%%%%%%%%%%%%%%%%%%%%%%%%%%%%%%%%%%%%%%%%%%%%%%%%%%%%%%%%%%%%%%%%%%%%%%%%
\section{Telefónny systém}  
Naozaj len v krátkosti. 
  
%%%%%%%%%%%%%%%%%%%%%%%%%%%%%%%%%%%%%%%
\subsection{Modemy}      
%%%%%%%%%%%%%%%%%%%%%%%%%%%%%%%%%%%%%%%
\subsection{Štandardy}    
%%%%%%%%%%%%%%%%%%%%%%%%%%%%%%%%%%%%%%%
\subsection{Konštelačné vzorky}  
%%%%%%%%%%%%%%%%%%%%%%%%%%%%%%%%%%%%%%%
\subsection{Modulačné techniky}   
%%%%%%%%%%%%%%%%%%%%%%%%%%%%%%%%%%%%%%%
\subsection{ISDN - systém pre domáce a firemné využitie}  
%%%%%%%%%%%%%%%%%%%%%%%%%%%%%%%%%%%%%%%
\subsection{xDSL, B-ISDN}       



%%%%%%%%%%%%%%%%%%%%%%%%%%%%%%%%%%%%%%%%%%%%%%%%%%%%%%%%%%%%%%%%%%%%%%%%%%%%%%%%
\section{Diaľkové vedenia a multiplexovanie, optické siete}
Tiež len v krátkosti. 

%%%%%%%%%%%%%%%%%%%%%%%%%%%%%%%%%%%%%%%
\subsection{FDMA/TDMA/CDMA}                           
%%%%%%%%%%%%%%%%%%%%%%%%%%%%%%%%%%%%%%%
\subsection{Synchrónne optické siete (SDH, SONET architektúra - definícia rámcov v SDH)}



\section{Referencie a odporúčaná literatúra}
\begin{itemize}                                
\item \url{http://en.wikipedia.org} 
\item \url{http://netlab.dcs.fmph.uniba.sk/siete/} 
\item \url{http://www.dcs.fmph.uniba.sk/\~plachetk/TEACHING/DISTRSYS2012/siete.pdf} 
\end{itemize}

\section{Ďalšie pojmy}
\begin{itemize}
\item lip synch (VoIP)                                  
\item \emph{ATM} is a core protocol used over the SONET/SDH backbone of the public switched telephone network (PSTN) and Integrated Services Digital Network (ISDN), but its use is declining in favour of All IP. 
\end{itemize}

\end{document}
