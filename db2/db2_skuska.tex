\documentclass[12pt,a4paper]{article}

\usepackage{listings}
\usepackage{graphicx}
\usepackage{tabularx} 

\lstset{
language=sh
,breaklines=true
,basicstyle=\ttfamily
, showstringspaces=false}

\author{Peter Csiba}
\textwidth 6.5in
\oddsidemargin 0.0in
\evensidemargin 0.0in

\title{Database systems 2 - Final}
\date{09-May-2014}
\author{Peter Csiba, petherz@gmail.com}

\begin{document}
\maketitle

\section{Introduction}
\url{http://infolab.stanford.edu/~ullman/cs345-notes.html}
 
 }

\section{Uvodna prednaska}

\subsection{Organizacia kurzu, system hodnotenia}

\subsection{Odporucana literatura}

\subsection{Moderne trendy v databazovych technologiach}

\subsection{Funkcne symboly v Datalogu, suvis s XML}

\subsection{Termy, Herbrandove univerzum, substitucie, matching, unifikacia}

\subsection{Preklad Datalogu s funkcnymi symbolmi do relacnej algebry}

\subsection{Operatory atov a vtoa}

\subsection{Naivna a seminaivna iteracia programov bez negacie}

\section{Negacia v databazach a logickom programovani}

\subsection{Logika, princip vylucenia tretieho, protirecenia}

\subsection{Formalne logicke systemy: propozicionalny kalkul, predikatovy kalkul, ...}

\subsection{Teorie a modely}

\subsection{Domenovo nezavisle formuly}

\subsection{Vypocet najmensieho modelu programu bez negacie (semi-) naivnou evaluaciou}

\subsection{Tarskeho veta o pevnom bode}

\subsection{Problem naivnej evaluacie programov s negaciou: minimalny pevny bod nemusi existovat}

\subsection{Stratifikovane a lokalne stratifikovane programy}

\subsection{Stabilne modely}

\subsection{Stabilne modely, Gelfond-Lifschitzova transformacia}

\subsection{Trojhodnotove modely, well-founded model}

\subsection{Metody vypoctu well-founded modelu: maximal unfounded set, alternujuci pevny bod}

\subsection{Hierarchia semantik, priklady inych semantik}

\subsection{Rozsirenie relacnej algebry o semijoin a antijoin}

\subsection{Vstupne mnoziny, pseudokluce, bezpecnost programov so zabudovanymi (napr. aritmetickymi) predikatmi}

\subsection{Vypocet programov s negaciou zdola nahor (Datalog) a zhora nadol (Prolog, SLD rezolucia)}

\section{Vypocet logickych programov}

\subsection{Datalog versus Prolog: vypocet zdola nahor vs zhora nadol}

\subsection{Rule-Goal Tree (RGT)}

\subsection{Sirenie vazieb a vysledkov: magicke a pomocne predikaty}

\subsection{Vypocet RGT do hlbky: expand_goal, expand_rule}

\subsection{Vypocet RGT do sirky: Queue-based RGT (QRGT)}

\subsection{Ozdoby predikatov a pravidiel, Rule-Goal Graph (RGG)}

\subsection{Usporiadanie podcielov v pravidle, zuniformnenie ozdob}

\subsection{Rektifikacia programu}

\subsection{Zabudovane podciele, dovolene ozdoby, realizovatelnost RGG}

\subsection{Metody vypoctu well-founded modelu: maximal unfounded set, alternujuci pevny bod}

\subsection{Hierarchia semantik, priklady inych semantik}

\subsection{Rozsirenie relacnej algebry o semijoin a antijoin}

\subsection{Vstupne mnoziny, pseudokluce, bezpecnost programov so zabudovanymi (napr. aritmetickymi) predikatmi}

\subsection{Vypocet programov s negaciou zdola nahor (Datalog) a zhora nadol (Prolog, SLD rezolucia)}

\section{Magicke transformacie}

\subsection{Vzory toku dat}

\subsection{Jednoducha magicka transformacia}

\subsection{Zovseobecnena magicka transformacia}

\subsection{Vypocet zovseobecnenych magickych programov}

\subsection{Dosledna rektifikacia: rozbalovanie cyklov}

\subsection{Problem s negaciou}

\section{Optimalizacie na urovni relacnej algebry}

\subsection{Reprezentacie algebraickych vyrazov: stromy, dagy, hypergrafy}

\subsection{Reprezentacia hypergrafu}

\subsection{GYO redukcia hypergrafu}

\subsection{Trhanie usi hypergrafu}

\subsection{Zakony relacnej algebry}

\subsection{Pravidla pre konstrukciu planu vypoctu v relacnej algebre}

\subsection{Optimalizacia poradia joinov}

\subsection{Eliminacia spolocnych podvyrazov}

\subsection{Semantika SQL}

\subsection{Minimalizacia operatorov relacnej algebry}

\subsection{Predpoklady na vypoctovy model, uskutocnitelnost operacii}

\section{Optimalizacie zalozene na redukcii hypergrafu}

\subsection{Reprezentacia dotazu hypergrafom}

\subsection{Transformacia dotazu na standardny hypergraf}

\subsection{Wong-Youssefiho algoritmus, preferencie na poradie ostranovanych hran}

\subsection{Uplny reduktor}

\subsection{Yannakakisov algoritmus}

\subsection{Hypergrafy a nekonecne relacie}

\section{Konjunktivne dotazy}

\subsection{Subsumpcia (pohltenie)}

\subsection{Pohlcujuce zobrazenia}

\subsection{Petrifikovane dotazy}

\subsection{Petrifikovane dotazy}

\subsection{Kanonicke databazy}

\subsection{Hladanie pohlteni, Sariayov algoritmus}

\subsection{Test pohltenia zjednoteni konjunktivnych dotazov}

\subsection{Pohltenie konjunktivneho dotazu programom a opacne}

\subsection{Pohltenie dotazov s negaciou, Levy-Sagivov test}

\subsection{Pohltenie dotazov so zabudovanymi predikatmi, Gupta-Zhang-Ozsoyoglu test}

\section{Minimalizacia konjunktivnych dotazov}

\subsection{Predpoklad univerzalnej relacie}

\subsection{Slabe pohltenie, slaba ekvivalencia konjunktivnych dotazov}

\subsection{Tableaux, konstrukcia tableaux}

\subsection{Minimalizacia tableaux}

\subsection{Uzatvaracia procedura (chase): vynutenie funkcnych zavislosti, multizavislosti, inkluznych zavislosti a joinovacich zavislosti}

\subsection{Silna ekvivalencia konjunktivnych dotazov}

\section{Vyssie normalne formy}

\subsection{Multizavislosti, 4NF}

\subsection{Joinovacie zavislosti, 5NF}

\subsection{Inkluzne zavislosti}

\subsection{Poloformalna metoda "spravneho" navrhu databazy}


 

\end{document}
